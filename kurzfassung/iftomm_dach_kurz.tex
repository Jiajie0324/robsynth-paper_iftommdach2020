% !TEX encoding = UTF-8 Unicode
% !TeX spellcheck = de_DE

% Latex-Template für die VDI Mechatroniktagung 2015
% Lehrstuhl für Regelungssystemtechnik, TU Dortmund
% Letzte Änderung: 18.02.2014

\documentclass[fleqn,a4paper,10pt]{article}

% Packages
\usepackage{mathptmx} % Times Schriftart (auch in der Math-Umgebung)
\usepackage{ngerman,bibgerm}
\usepackage[utf8]{inputenc} % Deutsche Sonderzeichen/Umlaute gestatten
\usepackage[T1]{fontenc}
\usepackage[nooneline,bf,figurename=Bild,labelsep=quad,font=small]{caption} % Bildunterschrift linksbündig

\usepackage{multicol}

\usepackage{amsmath} % Zur Abbildung mathematischer Symbole und Formeln
\usepackage{amssymb} 
\newcommand{\bm}[1]{\mathbf{#1}}
\renewcommand{\Phi}[1]{\varPhi{#1}}
\newcommand{\transp}[0]{{\mathrm{T}}}
\usepackage[fleqn]{nccmath} % Ausrichtung von Gleichungen

\usepackage{graphicx}
\graphicspath{{./Bilder/}}
\usepackage{color}
\usepackage{geometry}

\usepackage{booktabs} % Erweiterte Tabellenumgebung
\usepackage{cite} % Erweitertes Zitiermöglichkeiten

\usepackage{titlesec} % Abstand vor und hinter Abschnittstiteln.

\usepackage{etoolbox} % Scriptsammlung. Z.B. horizontalen Abstand zwischen Literatureintrag und jeweiligem Label verändern.


% Seiteneinstellungen
\geometry{left=2.025cm, right=2cm, top=2.07cm, bottom=2.8cm} % Geringe Änderungen, um mit Word übereinzustimmen.
\parindent0cm
\columnsep5mm
\parskip 0pt   
\pagestyle{empty}                  % Kopf- und Fußzeile entfernen

% Abschnittsüberschriften
\captionsetup[table]{skip=-1.4ex} % Abstand zwischen Tabelle und Tabellenbeschriftung
\captionsetup[figure]{skip=5mm} % Abstand zwischen Bild und Bildunterschrift

% Paper-Titel
\titlespacing*{\section}{0pt}{6pt}{6pt}
\titlespacing*{\subsection}{0pt}{6pt}{6pt}
\titleformat{\section}{\normalfont\Large\bfseries}{\thesection\hskip 1cm}{0pt}{} % Abstand zwischen Titelnummer und Beschriftung.
\titleformat{\subsection}{\normalfont\large\bfseries}{\thesubsection\hskip 0.7cm}{0pt}{}

% Gleichungen
\setlength{\mathindent}{0.5cm} % Abstand von Gleichungen zum linken Rand.

% Tabelleneinstellungen
%\renewcommand{\arraystretch}{9.5} % Abstand zwischen Zeilen in einer Tabelle -> Steht in Vorlage, führt in Formel-Array zu sehr großem Abstand
\setlength{\tabcolsep}{6pt} % Abstand zwischen Spalten in einer Tabelle

% Umgebungsdefinitionen
\newenvironment{mytitle}{\fontsize{16pt}{1.0} \selectfont \bfseries}{\\} 
\newenvironment*{myabstract}{\begin{Large}\bfseries}{\end{Large}\\[6pt]}%

\renewenvironment{figure}
  {\par\vspace{6pt}\noindent\minipage{\linewidth}}
  {\endminipage\par\vspace{6pt}}


\renewenvironment{table}
  {\par\vspace{6pt}\noindent\minipage{\linewidth}\fontsize{8.5pt}{1.2} \selectfont}
  {\endminipage\par\vspace{6pt}}

\makeatletter

\g@addto@macro\normalsize{%        % Abstand vor und hinter Gleichungsumgebungen.
  \setlength\abovedisplayskip{6pt}
  \setlength\belowdisplayskip{6pt}
  \setlength\abovedisplayshortskip{6pt}
  \setlength\belowdisplayshortskip{6pt}
}

\makeatother

% Literaturverzeichnis
\patchcmd{\thebibliography}{\advance\leftmargin\labelsep}
  {\advance\leftmargin\labelsep \labelsep=0.4cm}{}{} % Horizontalen Abstand anpassen (etoolbox)
\patchcmd{\thebibliography}{\section*}{\section}{}{} % Abschnittsnummer hinzufügen

\let\OLDthebibliography\thebibliography % Vertikalen Abstand anpassen
\renewcommand\thebibliography[1]{
  \OLDthebibliography{#1}
  \setlength{\parskip}{0pt}
  \setlength{\itemsep}{2pt}
}


\begin{document}

\begin{mytitle} 
Kombinierte Struktur- und Maßsynthese für Parallele Roboter: Dynamik mit vollständigen kinematischen Zwangsbedingungen \\[6pt] % Falls Zeilenumbruch erforderlich, \\[6pt] anfügen.
Combined Structural and Dimensional Synthesis of Parallel Robots: Dynamics using Full Kinematic Constraints
\end{mytitle}

Moritz Schappler, Prof. Dr.-Ing. Tobias Ortmaier, Leibniz Universität Hannover, Institut für Mechatronische Systeme, Appelstraße 11a, 30167 Hannover, Deutschland. Korrespondenz: moritz.schappler@imes.uni-hannover.de.

\vspace{24pt} % Bitte nicht entfernen.

\begin{myabstract} Kurzfassung \end{myabstract}
Hier steht auf einer Seite (Times, 10pt), warum der geneigte Leser sich unbedingt mit diesem Beitrag befassen sollte. Literaturzitate sollten entsprechend \cite{literaturpabst2010} in den Text gesetzt werden. Bei mehreren Literaturzitaten kann z.B. als \cite{literaturpabst2010,mueller2011} oder \cite{literaturpabst2010,meyer2009,mueller2011} zitiert werden.

\begin{figure}
	\centering
    \input{./Bilder/Optimierung_Uebersicht.pdf_tex}
    \captionof{figure}{Überblick.}
    \label{fig:overview}
\end{figure}

Hier geht der Text weiter. Ist das fett? $\mathit{{q}}$, $r$

$\chi_M(x)=\left\{\begin{array}{ll} 1, & x\in M \\0, & x\not\in M\end{array}\right.$
    
\begin{equation}
\chi_M(x)=\left\{\begin{array}{ll} 1, & x\in M \\
0, & x\not\in M\end{array}\right. .
\end{equation}
$\bm{\varPhi}$

$\tilde{\bm{J}}$

Und noch ein Bild

\begin{figure}
    \centering
    \input{./Bilder/Details_Kin_Dyn.pdf_tex}
    \captionof{figure}{Details für Kinematik und Dynamik Ansatz.}
    \label{fig:details_kindyn}
\end{figure}

\vspace{18pt} 

\begingroup\Large{\textbf{Literatur}}\endgroup
\begingroup % Bitte nicht entfernen.
\renewcommand{\section}[1]{} % Bitte nicht entfernen.
\begin{thebibliography}{99}
\bibitem{literaturpabst2010}
Literaturpabst, J. P:
\newblock \textit{Ein Literaturbeispiel aus einer Fachzeitschrift}.
\newblock MECHATRONIK. (2010) 1-2, ISSN 1867-2590, S. 11-22.

\bibitem{mueller2011}
Müller, H.-H (Hrsg.):
\newblock \textit{Ein Literaturbeispiel aus einem Fachbuch}.
\newblock Berlin, Heidelberg, New York: Hüpfer-Verlag, 2011.

\bibitem{meyer2009}
Meyer, B.; Müller, P.:
\newblock \textit{Ein Literaturbeispiel aus einem Tagungsbeitrag}.
\newblock In: Tagung Mechatronik 2009. 2./13. Mai 2009 Wiesloch. VDI Wissensforum Düsseldorf 2009. ISBN 978-3-98-12624-5-2. S. 295-302. 

\end{thebibliography}
\endgroup % Bitte nicht entfernen.

\clearpage  % Verhindert automatischen Umbruch auf der letzten Seite. Bei Bedarf entfernen.

\end{document}