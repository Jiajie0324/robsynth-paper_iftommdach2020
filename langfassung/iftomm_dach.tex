% Latex-Template für die VDI Mechatroniktagung 2015
% Lehrstuhl für Regelungssystemtechnik, TU Dortmund
% Letzte Änderung: 18.02.2014

\documentclass[fleqn,a4paper,10pt]{article}

% Packages
\usepackage{mathptmx} % Times Schriftart (auch in der Math-Umgebung)
\usepackage{ngerman,bibgerm}
\usepackage[utf8]{inputenc} % Deutsche Sonderzeichen/Umlaute gestatten
\usepackage[T1]{fontenc}
\usepackage[nooneline,bf,figurename=Bild,labelsep=quad,font=small]{caption} % Bildunterschrift linksbündig

\usepackage{multicol}

\usepackage{amsmath} % Zur Abbildung mathematischer Symbole und Formeln
\usepackage{amssymb} 
\newcommand{\bm}[1]{\mathbf{#1}}
\renewcommand{\Phi}[1]{\varPhi{#1}}
\newcommand{\transp}[0]{{\mathrm{T}}}
\usepackage[fleqn]{nccmath} % Ausrichtung von Gleichungen

\usepackage{graphicx}
\graphicspath{{../kurzfassung/Bilder/}}
\usepackage{color}
\usepackage{geometry}

\usepackage{booktabs} % Erweiterte Tabellenumgebung
\usepackage{cite} % Erweitertes Zitiermöglichkeiten

\usepackage{titlesec} % Abstand vor und hinter Abschnittstiteln.

\usepackage{etoolbox} % Scriptsammlung. Z.B. horizontalen Abstand zwischen Literatureintrag und jeweiligem Label verändern.


% Seiteneinstellungen
\geometry{left=2.025cm, right=2cm, top=2.07cm, bottom=2.8cm} % Geringe Änderungen, um mit Word übereinzustimmen.
\parindent0cm
\columnsep5mm
\parskip 0pt   
\pagestyle{empty}                  % Kopf- und Fußzeile entfernen

% Abschnittsüberschriften
\captionsetup[table]{skip=-1.4ex} % Abstand zwischen Tabelle und Tabellenbeschriftung
\captionsetup[figure]{skip=5mm} % Abstand zwischen Bild und Bildunterschrift

% Paper-Titel
\titlespacing*{\section}{0pt}{6pt}{6pt}
\titlespacing*{\subsection}{0pt}{6pt}{6pt}
\titleformat{\section}{\normalfont\Large\bfseries}{\thesection\hskip 1cm}{0pt}{} % Abstand zwischen Titelnummer und Beschriftung.
\titleformat{\subsection}{\normalfont\large\bfseries}{\thesubsection\hskip 0.7cm}{0pt}{}

% Gleichungen
\setlength{\mathindent}{0.5cm} % Abstand von Gleichungen zum linken Rand.

% Tabelleneinstellungen
%\renewcommand{\arraystretch}{9.5} % Abstand zwischen Zeilen in einer Tabelle -> Steht in Vorlage, führt in Formel-Array zu sehr großem Abstand
\setlength{\tabcolsep}{6pt} % Abstand zwischen Spalten in einer Tabelle

% Umgebungsdefinitionen
\newenvironment{mytitle}{\fontsize{16pt}{1.0} \selectfont \bfseries}{\\} 
\newenvironment*{myabstract}{\begin{Large}\bfseries}{\end{Large}\\[6pt]}%

\renewenvironment{figure}
  {\par\vspace{6pt}\noindent\minipage{\linewidth}}
  {\endminipage\par\vspace{6pt}}


\renewenvironment{table}
  {\par\vspace{6pt}\noindent\minipage{\linewidth}\fontsize{8.5pt}{1.2} \selectfont}
  {\endminipage\par\vspace{6pt}}

\makeatletter

\g@addto@macro\normalsize{%        % Abstand vor und hinter Gleichungsumgebungen.
  \setlength\abovedisplayskip{6pt}
  \setlength\belowdisplayskip{6pt}
  \setlength\abovedisplayshortskip{6pt}
  \setlength\belowdisplayshortskip{6pt}
}

\makeatother

% Literaturverzeichnis
\patchcmd{\thebibliography}{\advance\leftmargin\labelsep}
  {\advance\leftmargin\labelsep \labelsep=0.4cm}{}{} % Horizontalen Abstand anpassen (etoolbox)
\patchcmd{\thebibliography}{\section*}{\section}{}{} % Abschnittsnummer hinzufügen

\let\OLDthebibliography\thebibliography % Vertikalen Abstand anpassen
\renewcommand\thebibliography[1]{
  \OLDthebibliography{#1}
  \setlength{\parskip}{0pt}
  \setlength{\itemsep}{2pt}
}


\begin{document}

\begin{mytitle} 
Maßsynthese für Parallele Roboter: Einheitliche Kinematik und\\ Dynamik mit vollständigen kinematischen Zwangsbedingungen \\[6pt] % Falls Zeilenumbruch erforderlich, \\[6pt] anfügen.
Dimensional Synthesis of Parallel Robots: Unified Kinematics and\\ Dynamics using Full Kinematic Constraints
\end{mytitle}

Moritz Schappler, Prof. Dr.-Ing. Tobias Ortmaier, Leibniz Universität Hannover, Institut für Mechatronische Systeme, Appelstraße 11a, 30167 Hannover, Deutschland. Korrespondenz: moritz.schappler@imes.uni-hannover.de.

\vspace{24pt} % Bitte nicht entfernen.

\begin{myabstract} Kurzfassung \end{myabstract}
Hier steht auf Deutsch in maximal 10 Zeilen, warum der geneigte Leser sich unbedingt mit diesem Beitrag befassen sollte. Hier steht in maximal 10 Zeilen, warum der geneigte Leser sich unbedingt mit diesem Beitrag befassen sollte. Hier steht in maximal 10 Zeilen, warum der geneigte Leser sich unbedingt mit diesem Beitrag befassen sollte. Hier steht in maximal 10 Zeilen, warum der geneigte Leser sich unbedingt mit diesem Beitrag befassen sollte. Hier steht in maximal 10 Zeilen, warum der geneigte Leser sich unbedingt mit diesem Beitrag befassen sollte. Hier steht in maximal 10 Zeilen, warum der geneigte Leser sich unbedingt mit diesem Beitrag befassen sollte. Hier steht in maximal 10 Zeilen, warum der geneigte Leser sich unbedingt mit diesem Beitrag befassen sollte. Hier steht in maximal 10 Zeilen, warum der geneigte Leser sich unbedingt mit diesem Beitrag befassen sollte. Hier steht in maximal 10 Zeilen, warum der geneigte Leser sich unbedingt mit diesem Beitrag befassen sollte. Hier steht in maximal 10 Zeilen, warum der geneigte Leser sich unbedingt mit diesem Beitrag befassen sollte.


\vspace{12pt} % Bitte nicht entfernen.


\begin{myabstract} Abstract \end{myabstract}
Please explain in English at most in 10 lines, why anybody should spend his time reading your complete article. Please explain in English at most in 10 lines, why anybody should spend his time reading your complete article. Please explain in English at most in 10 lines, why anybody should spend his time reading your complete article. Please explain in English at most in 10 lines, why anybody should spend his time reading your complete article. Please explain in English at most in 10 lines, why anybody should spend his time reading your complete article. Please explain in English at most in 10 lines, why anybody should spend his time reading your complete article. Please explain in English at most in 10 lines, why anybody should spend his time reading your complete article. Please explain in English at most in 10 lines, why anybody should spend his time reading your complete article. Please explain in English at most in 10 lines, why anybody should spend his time reading your complete article. Please explain in English at most in 10 lines, why anybody should spend his time reading your complete article.


\vspace{18pt} % Bitte nicht entfernen.


\begin{multicols}{2}

\section{Einleitung}

Die Struktursynthese paralleler Roboter hat durch die systematische Nutzung der Schraubentheorie \cite{KongGos2007} und der evolutionären Morphologie \cite{Gogu2008} eine Vielzahl neuer Parallelkinematiken hervorgebracht, für die Großteils noch keine praktische Anwendung gefunden wurde.
Um für gegebene Anforderungen an einen Roboter den bestgeeignetsten auszuwählen, ist eine automatische Auslegung (Maßsynthese) notwendig, die für die Vielzahl verfügbarer Kinematiken anwendbar ist.
Der vorliegende Beitrag behandelt eine allgemeine Methodik für die Maßsynthese paralleler Kinematiken unter Ausnutzung struktureller Eigenschaften von Kinematik und Dynamik sowie der effizienten Strukturierung des Optimierungsproblems.

\section{Optimierungsproblem}

%Neuer Absatz: Detail zu Optimierung
%\begin{itemize}
%\item Die etablierte Partikel-Schwarm-Optimierung zur Lösung des hochdimensionalen Problems der Parameterfindung von Robotern wird hinsichtlich der Behandlung der Nebenbedingungen modifiziert.
%\item Stückweise definierte Fitness-Funktion mit Zielfunktion und Nebenbedingungen als Strafterm hierarchisch, um Rechenzeit zu sparen. Siehe Bild\,\ref{fig:uebersicht}.
%\item Energieverbrauch als Zielfunktion, da skalare Größe gut vergleichbar und andere Zielfunktionen wie Gesamtmasse und damit Dynamik dazu korrelierend
%\end{itemize}
Für das hochdimensionale Problem der Parameterfindung der Roboter wird die etablierte Partikel-Schwarm-Optimierung (PSO) eingesetzt, wie in Bild\,\ref{fig:uebersicht} skizziert.
Zur effizienten Berücksichtigung von Nebenbedingungen werden diese hierarchisch berechnet und normierte Strafterme $s_i$ direkt in der stückweise definierten Fitness-Funktion $f$ eingesetzt.
Als skalare Zielfunktion $z$ wird der Energieverbrauch des Roboters über eine Referenz-Trajektorie $\bm{x}(t)$ gewählt, was einen direkten Vergleich unterschiedlicher Kinematiken erlaubt.
Des Weiteren wird eine Korrelation mit anderen Zielfunktionen wie Gesamtmasse des Systems oder der Gleichförmigkeit der Eigenschaften (Konditionszahl der Jacobi-Matrix) erwartet.
Die hierarchische Eigenschaft wird durch die Bedingung $s_1(\bm{p})>s_2(\bm{p})>\dots>z(\bm{p})$ und eine Normierung umgesetzt.

\begin{figure*}
	\centering
	\input{../kurzfassung/Bilder/Optimierung_Uebersicht.pdf_tex}
	\captionof{figure}{Überblick über die PSO-Optimierung und die Struktur der Fitness-Funktion mit hierarchischen Nebenbedingungen.}
	\label{fig:uebersicht}
\end{figure*}

%Neuer Absatz: Kinematik und Dynamik
%\begin{itemize}
%\item Generische Ansätze für inverse Dynamik \cite{DoThanhKotHeiOrt2009b,AbdellatifHei2009} bisher explizit nur für klassische PKM definiert, deren Plattform mit Kugelgelenken verbunden ist
%\item Durch Nutzung von vollständigen Zwangsbedingungen (Position und Orientierung) \cite{Gogu2008,SchapplerTapOrt2019c} einheitlicher Ansatz zum Lösen der inversen Kinematik und der inversen Dynamik möglich.
%\item Durch die Reduzierung des Rechenaufwands ist die Auswertung vieler Parameterkombinationen möglich
%\end{itemize}

Zur Bestimmung von Nebenbedingungen wie der Einhaltung der Gelenkwinkelgrenzen ist eine Berechnung der passiven Gelenke und der Plattform-Koppelgelenke notwendig.
%Eine analytische Elimination der passiven Gelenke ist nicht für alle automatisch synthetisierten Strukturen aus möglich.
Zusätzlich wird ein effizienter und allgemeingültiger Ansatz für die Dynamik benötigt.
Der in Bild\,\ref{fig:details_kindyn} dargestellte und im Folgenden skizzierte  Ansatz erfüllt diese Anforderungen.

\section{Einheitliche Kinematik und Dynamik}

\begin{figure*}
	\centering
	\input{../kurzfassung/Bilder/Details_Kin_Dyn.pdf_tex}
	\captionof{figure}{Einheitliche Berechnung von inverser Kinematik und inverser Dynamik in der PSO-Fitness-Funktion als Ablaufskizze.}
	\label{fig:details_kindyn}
\end{figure*}

Zunächst werden die Gelenkgrößen $\bm{q},\dot{\bm{q}},\ddot{\bm{q}}$ des Roboters mittels der vollständigen \emph{inversen Kinematik} berechnet (linker Block in Bild\,\ref{fig:details_kindyn}).
Die dafür notwendigen vollständigen kinematischen Zwangsbedingungen (\glqq{}ZB\grqq{}) $\bm{\Phi}$ enthalten für jede Beinkette sowohl den Positions-, als auch den in Euler-Winkeln ausgedrückten Orientierungsfehler \cite{SchapplerTapOrt2019c}.
Durch Bildung der Gradienten ($\bm{\Phi}_\bm{q}$ und $\bm{\Phi}_\bm{x}$) können mit der vollständigen Jacobi-Matrix $\tilde{\bm{J}}$ neben den aktiven Gelenkkoordinaten $\bm{q}_\mathrm{a}$ auch passive Gelenke und Schnittgelenke der Plattform berechnet werden, die gemeinsam in $\bm{q}$ zusammengefasst sind.

Für die Bestimmung des Energieverbrauchs für eine Referenztrajektorie wird ein allgemeingültiger Ansatz für die \emph{inverse Dynamik} beliebiger paralleler Roboter benötigt.
Das zu diesem Zweck in \cite{AbdellatifHei2009,DoThanhKotHeiOrt2009b} vorgestellte Prinzip der Energieäquivalenz wird dafür erweitert.
Zunächst erfolgt die Zerlegung des parallelen Roboters in einzelne Subsysteme (Beinketten und Plattform), in deren jeweiligen Minimalkoordinaten (Gelenk- und Plattformkoordinaten) die Dynamik einzeln nach dem \textsc{Lagrange}schen Formalismus aufgestellt wird (vollständige Massenmatrix $\tilde{\bm{M}}$, Coriolis- und Fliehkraftterme $\tilde{\bm{c}}$, Gravitationsterme $\tilde{\bm{g}}$) \cite{DoThanhKotHeiOrt2009b}.
Anschließend erfolgt die Projektion der Dynamik in die Minimal-/Plattformkoordinaten $\bm{x}$ des Roboters durch Anwendung des \textsc{D'Alembert}schen Prinzips, woraus die Dynamik-Terme $\bm{M},\bm{c},\bm{g}$ bezogen auf die Plattform resultieren \cite{DoThanhKotHeiOrt2009b,AbdellatifHei2009} (siehe mittlerer Block in Bild\,\ref{fig:details_kindyn}).
Die Jacobi-Matrix $\tilde{\bm{J}}$ aus der inversen Kinematik kann direkt ohne Neuberechnung für die Dynamik-Projektionsmatrix $\tilde{\bm{R}}$ eingesetzt werden.
Durch die Verwendung rotatorischer Zwangsbedingungen \cite{SchapplerTapOrt2019c} zusätzlich zu den translatorischen Zwangsbedingungen kann die implizite Annahme aus \cite{AbdellatifHei2009} aufgehoben werden, dass die Beinketten räumlicher Parallelkinematiken in Kugelgelenken enden müssen.
Diese Bedingung wird z.\,B. von einigen der durch Gosselin et al. \cite{KongGos2007} und Gogu et al. \cite{Gogu2008} erzeugten Roboter nicht erfüllt.

Für die Berechnung der \emph{Energie als Zielfunktion} (rechter Block in Bild\,\ref{fig:details_kindyn}) erfolgt eine Projektion der Dynamik auf die Antriebskräfte $\bm{\tau}_\mathrm{a}$. 
Für die dafür notwendige Jacobi-Matrix $\bm{J}$ müssen lediglich die entsprechenden Spalten aus $\tilde{\bm{J}}$ gewählt werden.
Es wird angenommen, dass alle Antriebe ihre Leistung in einem elektrischen Zwischenkreis austauschen können (resultierend zu $P_\mathrm{Kreis}$), was einem zukünftigen Szenario zur Erhöhung der Energieeffizienz entspricht.
Für die aus dem Netz aufzunehmende Leistung $P_\mathrm{Netz}$ wird die Möglichkeit einer  Rückspeisung ausgeschlossen.
Aus der aufintegrierten Energie $E_\mathrm{Netz}$ wird durch eine Normierung $f_\mathrm{norm}(\cdot)$ (mittels arctan-Funktion) der Zielfunktionswert $z$ gebildet.

\section{Ergebnisse}

Erste Ergebnisse der Methodik sind beispielhaft in Bild\,\ref{fig:ergebnisse} dargestellt.
Es wurde eine Optimierung der 6\underline{P}RRS- und  6\underline{R}RRS-Kinematik durchgeführt, wobei das Kugelgelenk (S) einmal beibehalten und einmal durch drei versetzt angeordnete Drehgelenke (RRR) ersetzt wurde, um die allgemeinere Dynamik-Modellierung in der Maßsynthese zu validieren.
Die der Optimierung zugrunde liegende Aufgabe ist das Abfahren der Eckpunkte eines Würfels mit Kantenlänge $30\,\textrm{mm}$ und Kippwinkeln bis zu $10^\circ$ mit einem Beschleunigungs-Trapez-Profil mit einer Gesamt-Verfahrdauer von ca. $11\,\textrm{s}$.

\begin{figure*}
\centering
\input{../kurzfassung/Bilder/Optimierung_Ergebnisse.pdf_tex}
\captionof{figure}{Ergebnisse der Maßsynthese für vier Beispiel-PKM mit Angabe der Zielfunktion $E_\mathrm{Netz}$ und der Parameteranzahl $\mathrm{dim}(\bm{p})$.}
\label{fig:ergebnisse}
\end{figure*}

%Durch die Vorgabe einer maximalen Plattform-Geschwindigkeit von $1\,\textrm{m/s}$ bzw. $60^\circ\textrm{/s}$ und -Beschleunigung von $3\,\textrm{m/s}^2$ bzw. $180^\circ\textrm{/s}^2$ und einer Zusatzmasse von $3\,\textrm{kg}$ wird versucht, einen ausreichend großen Einfluss der Dynamik auf das Optimierungsergebnis zu erzielen.
Die Optimierungsparameter der durchgeführten Maßsynthese sind u.\,a. die Größenskalierung des Roboters, die \textsc{Denavit-Hartenberg}-Parameter der seriellen Beinketten, die Durchmesser von Gestell und Plattform und Morphologieparameter wie Koppelpunktpaarabstand und Neigung des gestellnahen Gelenks.
%Über die Modellierung der Beinsegmente als Hohlzylinder und der Plattform als Kreisscheibe führt die Veränderung der kinematikbeeinflussenden Parameter
%Parameter wie Gelenk- und Antriebsauswahl, Stärke von Beinsegmenten und Plattform wurden nicht optimiert sondern in späteren Untersuchungen als Entwurfsoptimierung separat betrachtet.
%Einfluss auf die Ergebnisse hatten daher nur die Modellierung der Beinsegmente als Hohlzylinder und der Plattform als Kreisscheibe.
Der PSO-Algorithmus wurde mit 30 Generationen und 70 Individuen parametriert, was angesichts der 15 bis 19 Optimierungsparameter noch gering erscheint.
Der hohen Dimensionalität wurde durch eine Reduzierung der Abhängigkeiten zwischen Parametern und plausibel gewählten Grenzen begegnet.









\section{Zusammenfassung}

In dem Konferenzvortrag werden die Ergebnisse und der Ansatz der Optimierung im Detail und weitere Beispiele vorgestellt.
%Dafür wird die Optimierung für weitere Kinematiken und längere Optimierungsdauern durchgeführt sowie statistischen Untersuchungen angestrebt.
Perspektivisch wird auch die Integration der Entwurfsoptimierung von Antriebs- und Segmentdimensionierung und die Behandlung zusätzlicher Nebenbedingungen und Zielfunktionen diskutiert.

%Als weitere Arbeiten sind Plausibilisierungen hinsichtlich Straftermen für Selbstkollisionen und die Reduktion der Optimierungsparameter geplant.

\section{Danksagung}

Die Autoren bedanken sich für die Unterstützung durch die Deutsche Forschungsgemeinschaft (DFG, OR 196/33-1).

\begin{thebibliography}{99}

\bibitem{KongGos2007}
Kong, X.; Gosselin, C.M.:
\newblock \textit{Type synthesis of parallel mechanisms}.
\newblock Springer-Verlag Berlin Heidelberg, 2007.

\bibitem{Gogu2008}
Gogu, G.:
\newblock \textit{Structural Synthesis of Parallel Robots, Part 1: Methodology}.
\newblock Springer Netherlands, 2008.

\bibitem{AbdellatifHei2009}
Abdellatif, H. and Heimann, B.:
\newblock \textit{Computational efficient inverse dynamics of 6-DOF fully parallel manipulators by using the Lagrangian formalism}.
\newblock Mechanism and Machine Theory 44. (2009), 1, S. 192-207.


\bibitem{DoThanhKotHeiOrt2009b}
Do Thanh, T.; Kotlarski, J.; Heimann, B.; Ortmaier, T.:
\newblock \textit{On the inverse dynamics problem of general parallel robots}.
\newblock In: IEEE International Conference on Mechatronics 2009. 14.-17. April 2009 Malaga/Spanien. IEEE 2009. ISBN 978-1-4244-4194-5. S. 1-6. 

\bibitem{SchapplerTapOrt2019c}
Schappler, M. and Tappe, S. and Ortmaier, T.:
\newblock \textit{Modeling Parallel Robot Kinematics for 3T2R and 3T3R Tasks Using Reciprocal Sets of Euler Angles}.
\newblock MDPI Robotics 8, 68 (2019).

\end{thebibliography}


\clearpage  % Verhindert automatischen Umbruch auf der letzten Seite. Bei Bedarf entfernen.


\end{multicols}


\end{document}